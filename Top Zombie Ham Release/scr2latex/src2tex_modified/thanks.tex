
\centerline{\bf Acknowledgement}

\bigskip

\noindent
The authors would like to express their gratitude to

Andrew Allen ANDA

Mitsuru AOYAMA

Youzou DEGUCHI

Yoriaki FUJIMORI

Shinichi HATAKI{\dag}

Norihiro ITOH

Yoh ITOH

Hiroshi KAI

Azuki KANEKO

Shin MAEDA\footnote{\dag}{one of the authors of versions 2.8 and 2.9}

Takashi NAKAYAMA

Naoto NIKI

Kiyoko NISHIZAWA

Sakuro OZAWA

Issei SUZUKI

Hitoshi UCHIDA

Kenji YAMADA

Masayuki YAMASAKI\footnote{\ddag}{one of the authors of versions 2.10 and 2.11}

\noindent
for giving them invaluable criticisms and suggestions.
